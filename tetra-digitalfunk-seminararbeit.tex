%% Vorlage Bachelorarbeit

%% Versionshistorie:

%% v1.0: Erstellung durch Johannes Woske, IT2010, j.woske+latex@gmail.com
%% v2.0: Überarbeitung und Ergänzung durch Anne Traulsen, IT2015, a.traulsen+latex@gmail.com

\documentclass[
	12pt, %Schriftgröße
	a4paper,
	liststotoc, %Inhaltsverzeichniseinträge für Listen (z.B. Abbildungen)
	bibtotoc, %Inhaltsverzeichniseinträge f+r Quellen
	pointlessnumbers, %Entfernt Punkt hinter Gliederungsnummern
	ngerman, %Sprachpaket
	headsepline, %Headertrennlinie
	%footsepline, %Footertrennlinie
	oneside %einseitiges Druckformat %%% Unterdrücken der leeren Seite nach Titelblatt
	]{scrbook} %Dokumentenklasse (Koma-Script)
\usepackage[T1]{fontenc}
\usepackage{float}
\usepackage[utf8]{inputenc}
\usepackage[ngerman]{babel}
\usepackage{url}
\usepackage{graphicx} %Bilder einfügen
%\usepackage{pdfpages} %PDF einfügen
\usepackage[a4paper, margin=1in]{geometry}
\usepackage[right]{eurosym} %Euro-Zeichen
\usepackage{amssymb}
\usepackage{cite} %Quellenangaben
\usepackage{setspace} % Zeilenabstand
\usepackage[ 
   colorlinks,        % Links ohne Umrandungen in zu wählender Farbe 
   linkcolor=black,   % Farbe interner Verweise 
   filecolor=black,   % Farbe externer Verweise 
   citecolor=black,   % Farbe von Zitaten 
   urlcolor=blue	  % Farbe von Links
   ]{hyperref} %Verlinkungen
\usepackage[figure]{hypcap}
\usepackage[ngerman]{translator}
\usepackage{blindtext} % Lorem-Ipsum-Plugin
\usepackage[acronym, nonumberlist]{glossaries} %% use after hyperref %Glossar-Paket laden
%\usepackage[
	%nonumberlist, %keine Seitenzahlen anzeigen
	%acronym,      %ein Abkürzungsverzeichnis erstellen
	%toc,          %Einträge im Inhaltsverzeichnis
	%section      %im Inhaltsverzeichnis auf section-Ebene erscheinen
	%]
%{glossaries}

\usepackage{listings,xcolor} %Codeanzeige
\usepackage[normalem]{ulem}
\useunder{\uline}{\ul}{}

\usepackage{chngcntr}
\counterwithout{figure}{chapter}
\counterwithout{table}{chapter}

\definecolor{dkgreen}{rgb}{0,.6,0}
\definecolor{dkblue}{rgb}{0,0,.6}
\definecolor{dkyellow}{cmyk}{0,0,.8,.3}

\lstset{
    numbers=left, 
    numberstyle=\tiny, 
    numbersep=5pt,
    breaklines=true,
    frame=single,
    escapeinside={(*@}{@*)}, %nicht anzuzeigende Ausdrücke, z.B. für Labels
    language=sh,
    basicstyle=\ttfamily\fontsize{10}{12}\selectfont,
    keywordstyle    = \color{dkblue},
    stringstyle     = \color{red},
    identifierstyle = \color{dkgreen},
    commentstyle    = \color{gray},
    emph            =[1]{php},
    emphstyle       =[1]\color{black},
    emph            =[2]{if,and,or,else},
    emphstyle       =[2]\color{dkyellow}
    } 

%%%%%%%%%%%%%%%%%%%%%%%%%%%%%%%%%%%%%%%%%%%%%%%%%%%%%
%%%%%%%%%%% Sonderformatierung
%%%%%%%%%%%%%%%%%%%%%%%%%%%%%%%%%%%%%%%%%%%%%%%%%%%%%

% Seitenabstände definieren
\geometry{verbose,tmargin=3cm,bmargin=2cm,lmargin=3cm,rmargin=3cm} 

% Hurenkinder und Schusterjungen verhindern (Ja, das heißt wirklich so!!!)
\clubpenalty = 10000 \widowpenalty = 10000 \displaywidowpenalty = 10000 

\newcommand{\footfigref}[1]{\footnote{Abb. \ref{#1} auf Seite \pageref{#1}}}

%% Bei Referenzen im Text wird jetzt bei allen Ebenen "Kapitel" vorgestellt, z.b. Kapitel 2, Kapitel 2.2, Kapitel 6.3.2
\addto\extrasngerman{%
    \def\sectionautorefname{Kapitel}%
    \def\subsectionautorefname{Kapitel}%
    \def\subsubsectionautorefname{Kapitel}%
    }

% Vertikaler Abstand zwischen Ende Textblock - Ende Fußzeile --> Abstand der Seitenzahl von Rand erhöhen 
\setlength{\footskip}{10mm}

% Abstand vor/nach Überschriften verändern

\RedeclareSectionCommand[%
    beforeskip=0.5\baselineskip,
    afterskip=0.5\baselineskip
]{chapter}

\RedeclareSectionCommand[%
    beforeskip=0.5\baselineskip,
    afterskip=0.5\baselineskip
]{section}

\RedeclareSectionCommand[%
    beforeskip=0.1\baselineskip,
    afterskip=0.1\baselineskip
]{subsection}

\RedeclareSectionCommand[%
    beforeskip=0.01\baselineskip,
    %%afterskip=0.2\baselineskip
]{paragraph}

\setlength{\abovecaptionskip}{4pt}  % 1pc=12pt 
\setlength{\belowcaptionskip}{0pt}
%\setlength{\textfloatsep}{4pt}
\setlength{\intextsep}{1pc}

%% Verkleinerung der Textgröße unter Abbildungen
\addtokomafont{caption}{\small}

% falsche Default-Silbentrennung überschreiben
\include{hyphenation}

% Den Punkt am Ende der Glossareinträge deaktivieren
\renewcommand*{\glspostdescription}{}

%Glossar-Befehle anschalten
%\makeglossaries

% sorgt dafür, dass bei Leerzeile die Einrückung verhindert und stattdessen eine Leerzeile eingefügt wird % erspart bigskips und erhöht die Lesbarkeit im LaTeX-Text 
\KOMAoptions{parskip=full*}

% ändert Titelschriftart in Serifen-Normalschriftart
\addtokomafont{disposition}{\rmfamily} 

\makenoidxglossaries

\loadglsentries{glossar.tex}


%%%%%%%%%%%%%%%%%%%%%%%%%%%%%%%%%%%%%%%%%%%%%%%%%%%%%
%%%%%%%%%%% Textbausteine
%%%%%%%%%%%%%%%%%%%%%%%%%%%%%%%%%%%%%%%%%%%%%%%%%%%%%
%%%%%%%%%%%% Studentenname
\newcommand{\studentName}{Finn Noel Valentin Margraf}
%%%%%%%%%%%% Typ der Arbeit
\newcommand{\type}{Hausarbeit}
%%%%%%%%%%%% Thema
\newcommand{\topic}{TETRA - Digitaler Bündelfunk in den Einheiten der BOS}
%%%%%%%%%%%% Untertitel
\newcommand{\subtopic}{Gegenüberstellung der Vorteile, Nachteile und Risiken}
%%%%%%%%%%%% Studienbereich
\newcommand{\studienbereich}{Technik}
%%%%%%%%%%%% Fachrichtung
\newcommand{\fachrichtung}{Informatik}
%%%%%%%%%%%% Studiengang
\newcommand{\studiengang}{Informatik}
%%%%%%%%%%%% Betrieb
\newcommand{\company}{Flughafen Berlin Brandenburg GmbH}
%%%%%%%%%%%% Betreuer HWR
\newcommand{\betreuerHS}{Mia Chelsea Pertubla Reyes}
%%%%%%%%%%%% Jahrgang
\newcommand{\jahrgang}{2025}
%%%%%%%%%%%%%%%%%%%%%%%%%%%%%%%%%%%%%%%%%%%%%%%%%%%%%>>>>>>>

\begin{document}

%%%%%%%%%%%%%%%%%%%%%%%%%%%%%%%%%%%%%%%%%%%%%%%%%%%%%>>>>>>>
%%%%%%%%%%% Titelblatt

%% Anordnung und Aussehen von Titel und Untertitel

\subject{\type}

\title{
\normalfont\endgraf\rule{\textwidth}{.4pt}
\begingroup
	\centering
	\linespread{1.5}
	\huge\topic
\endgroup
\linespread{1}
\ \\ % Falls kein Subtopic, auskommentieren
\ \\ % Falls kein Subtopic, auskommentieren
\large\subtopic % Falls kein Subtopic, auskommentieren
\endgraf\rule{\textwidth}{.4pt}
}
 
%%Eigentlich nicht besonders schön, aber Koma erlaubt die Anordnung eines weiteren Felden (hier: Fachbereich) nicht
\date{\normalsize vorgelegt am 17. Dezember 2025\\ \textbullet \\ Fachbereich Duales Studium Wirtschaft / Technik \\
Hochschule für Wirtschaft und Recht Berlin}
%% \date muss leer angegeben werden, um die Default-Datumsanzeige zu unterdrücken

\publishers{
	\begin{tabular}{l l}
	\textbf{\normalsize{}} & \normalsize{}  \tabularnewline
	\textbf{\normalsize{}} & \normalsize{}  \tabularnewline
	\textbf{\normalsize{Name:}} & \normalsize{\studentName}  \tabularnewline
	\textbf{\normalsize{Ausbildungsbetrieb:}} & \normalsize{\company}  \tabularnewline
	\textbf{\normalsize{Studienbereich:}} & \normalsize{\studienbereich}  \tabularnewline
	\textbf{\normalsize{Fachrichtung:}} & \normalsize{\fachrichtung} \tabularnewline
	\textbf{\normalsize{Studiengang:}} & \normalsize{\studiengang} \tabularnewline
	\textbf{\normalsize{Studienjahrgang:}} & \normalsize{\jahrgang} \tabularnewline
	\textbf{\normalsize{Erstgutachter:}} & \normalsize{\betreuerHS}
	%%\tabularnewline
	%%\textbf{\normalsize{Wörter:}} & \normalsize{XXXX}
	\end{tabular}
	}

\titlehead{\begin{center}
    \includegraphics[scale=0.7]{bilder/header_logo.PNG}
    \end{center}
    }

\maketitle

\onehalfspacing % anderthalbfacher Zeilenabstand

%%%%%%%%%%%%%%%%%%%%%%%%%%%%%%%%%%%%%%%%%%%%%%%%%%%%%%%%%%%%%%%%%%%%%%%%%%%%%%%%%%%%%%%%%%%%%%%%%%%%%%%%%%%%%%%%%%%%%%%%%%%
%%%%%%%%%%% Dokumenteninhalt START
%%%%%%%%%%%%%%%%%%%%%%%%%%%%%%%%%%%%%%%%%%%%%%%%%%%%%%%%%%%%%%%%%%%%%%%%%%%%%%%%%%%%%%%%%%%%%%%%%%%%%%%%%%%%%%%%%%%%%%%%%%%

%1 		Einleitung
%2-3 	TETRA Standard
%4-5		BOS-Digitalfunknetz
%6		Stand der Literatur
%7-8		Vor- und Nachteile
%9		Risiken
%10		Fazit


%%%%%%%%%%%%%%%%%%%%%%%%%%%%%%%%%%%%%%%%%%%%%%%%%%%%%
%%%%%%%%%%% Abstract
\chapter*{Abstract}
\addcontentsline{toc}{chapter}{Abstract}

Die erste überarbeitete Auflage der Bachelorarbeit-Vorlage bietet einige Neuerungen, die in \autoref{sec:allgemein} näher erläutert werden.

Wenn die Abstract-Seite nicht die zweite Seite im Dokument ist, ist der Titel zu lang ;)

\pagenumbering{Roman} % römische Seitenzahlen

%%%%%%%%%%%%%%%%%%%%%%%%%%%%%%%%%%%%%%%%%%%%%%%%%%%%%
%%%%%%%%%%% Inhaltsverzeichnis, Tabellen, Abbildungen, etc.
\newpage

\tableofcontents{}
\addcontentsline{toc}{chapter}{Inhaltsverzeichnis}

\listoffigures

\section*{Hinweis}

Aus Gründen der besseren Lesbarkeit wird im Text verallgemeinernd die maskuline Form, sofern möglich, verwendet.
Diese Formulierungen umfassen gleichermaßen weibliche und männliche Personen.

\clearpage

\addcontentsline{toc}{chapter}{Akronyme}
\printnoidxglossaries

\clearpage

%% arabische Seitenzahlen
\pagenumbering{arabic}

%%%%%%%%%%%%%%%%%%%%%%%%%%%%%%%%%%%%%%%%%%%%%%%%%%%%%
%%%%%%%%%%% Einführung

\chapter{Einleitung}\label{ch:einleitung}

Laut einer Statistik des Deutschen Feuerwehrverbandes (DFV) aus dem Jahr 2022 kommen im Jahr rund 1,02 Millionen Einsätze
in den Bereichen Brandschutz, Technische Hilfeleistung und Tierrettung sowie 157.480 Fehlalarme zustande, Tendenz steigend. \cite[S. 2]{deutscherfeuerwehrverbandEinsatzeNachTatigkeitsbereichen2022}
Mit wachsenden Einsatzzahlen steigt auch die Anforderung an eine zuverlässige Kommunikation der Einsatzkräfte untereinander.
Mit der Einführung des digitalen Bündelfunks für alle \gls{BOS} auf \gls{TETRA}-Basis im Jahr 2015 \cite{bdbosBDBOSGeschichte} findet seither in Deutschland
eine Umstellung vom analogen Funknetz auf das digitale Funknetz statt.
Bereits im Jahr 2024 waren etwa 1,2 Millionen \gls{TETRA}-Endgeräte in Deutschland im Einsatz. \cite[S. 19]{bdbosFAQDigitalfunkBOS2024}

Trotz anhaltenden Wachstums steht das \gls{TETRA}-Netz allerdings auch immer wieder in der Kritik.
Insbesondere vergangene, einschlägige flächendeckende Ausfälle des Netzes, wie zuletzt am \date{06. Mai 2025} sorgen für Unmut. \cite[S. 1]{agbfdfvErkenntnisseAusGrossflachigen2025}

Aus diesem Grund beschäftigt sich die folgende Hausarbeit insbesondere mit den Vor- und Nachteilen des \gls{TETRA}-Digitalfunks
in Hinblick auf die Einheiten der \gls{BOS} in Deutschland.
Außerdem sollen mögliche Risiken und Gefahren, die mit der Nutzung des \gls{TETRA}-Netzes einhergehen, betrachtet werden.
Zum Abschluss soll außerdem die Frage beantwortet werden, ob der \gls{TETRA}-Digitalfunk den Anforderungen der \gls{BOS} gerecht wird.

\chapter{Theoretische Einführung}\label{ch:einfuehrung}
\section{Der TETRA-Standard}\label{sec:der-tetra-standard}

Der \gls{TETRA} Digitalfunkstandard ist ein von der \gls{ETSI} entwickelter Standard für die digitale Mobilfunkkommunikation,
der speziell für den Einsatz in professionellen und behördlichen Kommunikationssystemen konzipiert wurde. \cite{etsiTETRA}
Es handelt sich dabei um einen offenen Standard, der in seiner aktuellen Form seit 2001 existiert und in mehreren von der \gls{ETSI} definierten
Standardspezifikationen beschrieben ist.

Ganz fundamental wird zwischen dem \gls{TMO} (Netzbetrieb) und dem \gls{DMO} (Direktbetrieb) unterschieden.
Wobei der \gls{TMO} der Betrieb über eine Infrastruktur (Basisstationen, Vermittlungsstellen) erfolgt,
während der \gls{DMO} die direkte Kommunikation zwischen Endgeräten ohne Infrastruktur ermöglicht.

Für den \gls{TMO}-Betrieb spielt insbesondere die \gls{ETSI} EN 300 392-Reihe~\cite{tccaETSIStandards} eine wichtige Rolle, die die technischen Anforderungen und Spezifikationen für \gls{TETRA}-Netzwerke festlegt.
Hauptsächliche Grundlage bildet dabei die \gls{ETSI} EN 300 392-2 Spezifikation, die die physikalische Schicht und den Zugriff auf das Funkmedium definiert. \cite[S. 30]{etsiTerrestrialTrunkedRadio2001}
Für den \gls{DMO}-Betrieb ist die \gls{ETSI} EN 300 396~\cite{tccaETSIStandards} Reihe einschlägig, soll im weiteren Verlauf dieser Arbeit jedoch nicht weiter betrachtet werden.

Im Folgenden soll das fundamentale \gls{TDMA}-Verfahren des \gls{TETRA}-Standards näher betrachtet werden.

\subsection{Das TDMA-Verfahren im TETRA-Standard}\label{subsec:tdma-verfahren-tetra-standard}

\gls{TETRA} selbst ist ein zeitmultiplexbasiertes System, das auf dem \gls{TDMA}-Verfahren basiert.
Anstelle die Datenpakete kontinuierlich über die Funkfrequenz zu übertragen, werden die Daten in diskrete Zeitschlitze aufgeteilt.
Jeder Zeitschlitz kann von einem einzelnen Benutzer oder einer Gruppe von Benutzern genutzt werden, wodurch eine effiziente Nutzung der verfügbaren Bandbreite ermöglicht wird.
Man spricht daher auch von einem ``Bündelfunk'', da mehrere Kommunikationskanäle auf einer Frequenz gebündelt werden.

Im \gls{TETRA}-Standard sind pro Frequenzkanal vier Zeitschlitze definiert.
Diese vier Zeitschlitze werden zusammengefasst als ein \gls{TDMA} frame bezeichnet.
Jeder Zeitschlitz hat dabei eine Dauer von 14,167 ms und beinhaltet Platz für 510 bit an Daten, was zu einer Gesamtframedauer von rund 56,67 ms führt. \cite[S. 40f.]{etsiTerrestrialTrunkedRadio2001}
Neben den \gls{TDMA} frames sind auch noch weitere, übergeordnete Strukturen definiert, die sogenannten Multi- und Hyperframes.
Diese spielen jedoch im weiteren Verlauf dieser Arbeit keine Rolle und werden daher nicht weiter betrachtet.

\begin{figure}[h]
    \centering
    \includegraphics[scale=0.2]{bilder/TETRA_TDMA}
    \caption[TETRA TDAM Verfahren: Veranschaulichung]{Veranschaulichung des TETRA-TDMA-Verfahrens, eigene Abbildung nach \cite[S. 40]{etsiTerrestrialTrunkedRadio2001}}% \cite[S. 40]{ETSI_EN_300_392_2_V3_3_1}
    \label{fig:tetra-tdma}
\end{figure}

\newpage

\section{Das BOS-Digitalfunknetz}\label{sec:das-bos-digitalfunknetz}

Mit ca.\@ 5000 Basisstationen und einer Netzabdeckung von knapp 99,2\% ist das deutsche, von der \gls{BDBOS} Digitalfunknetz (kurz: \gls{BOS}-Digitalfunknetz) das größte weltweit.
Der Aufbau eines solchen Netzes ist dabei keineswegs trivial. \cite[S. 19]{bdbosFAQDigitalfunkBOS2024}

Grundlegend wird zwischen dem ``Kern-Netz'' und dem ``Funk- und Zugangsnetz'' unterschieden.

\subsection{Funk- und Zugangsnetz}\label{subsec:funk-und-zugangsnetz}

Das Funk- und Zugangsnetz ist die wesentlichen Schnittstellen zwischen den Endgeräten und der rückwärtigen Netzwerkinfrastruktur.
Es besteht im Wesentlichen aus den Basisstationen, die in ihren Funkzellen (Cells) die Netzabdeckung bereitstellen.
Eine Basisstation verwaltet dabei u.A.\@ die Zeitschlitze im \gls{TDMA}-Verfahren und stellt die Verbindung unter den Endgeräten und ihren Rufgruppen her. \cite[S. 20]{bdbosFAQDigitalfunkBOS2024}

\subsection{Kern-Netz}\label{subsec:kern-netz}

Das Kern-Netz stellt das Rückgrat des \gls{BOS}-Digitalfunknetzes dar.
Es besteht aus zwei wesentlichen Netzwerkschichten, die in Form von \glspl{DXT} und \glspl{DXTT} realisiert sind.

Eine \gls{DXT} (Vermittlungsstelle) ist dabei primäres Bindeglied zwischen den Basisstationen.
Sie verwaltet den Datenverkehr zwischen diesen und steuert Basisstation/Funkzellen übergreifende Kommunikationsvorgänge.
In Deutschland werden Stand 2025 64 \glspl{DXT} betrieben. \cite[S. 5]{bdbosWarumEsKein2025}

Eine \gls{DXTT} (Transitvermittlungsstelle) hingegen vermittelt den Datenverkehr zwischen verschiedenen \glspl{DXT}.
Sie ist somit für die übergeordnete und überregionale Steuerung und Verwaltung des gesamten Netzes zuständig. \cite[S. 18f.]{bdbosFAQDigitalfunkBOS2024}
Aktuell werden 4 \glspl{DXTT} in Deutschland betrieben. \cite[S. 5]{bdbosWarumEsKein2025}

\begin{figure}[h]
    \centering
    \includegraphics[scale=0.4]{bilder/BOS_AUFBAU}
    \caption[Aufbau BOS-Digitalfunknetz: Visualisierung]{(Geo-)Grafische Visualisierung des BOS-Digitalfunknetzes am Beispiel Sachsens, eigene Abbildung nach \cite[S. 19]{bdbosFAQDigitalfunkBOS2024}}
    \label{fig:bos-netz-aufbau}
\end{figure}

\subsection{Ende-zu-Ende-Verschlüsselung}\label{subsec:ende-zu-ende-verschlusselung}

Gerade im Umfeld der \gls{BOS} ist die Informationssicherheit von besonderer Bedeutung.
Aus diesem Grund ist eine verschlüsselte Kommunikation zwischen den Endgeräten im \gls{BOS}-Digitalfunknetz essenziell.
Der \gls{TETRA}-Standard beinhaltet hierfür standardmäßig eine Luftschnittstellenverschlüsselung, welche den reinen Übertragungsweg zwischen Endgerät und Basisstation schützt.
Um jedoch auch die Informationssicherheit über das gesamte Netz hinweg zu gewährleisten, führt das \gls{BOS}-Digitalfunknetz zusätzlich eine Ende-zu-Ende-Verschlüsselung ein.
Hierfür erhält jeder Teilnehmer eine kryptografische Sicherheitskarte, welche zum einen die Autorisierung im Netz durch eine eindeutige Teilnehmeradresse ermöglicht
und zum anderen die benötigten kryptografischen Schlüssel für die Ende-zu-Ende-Verschlüsselung bereitstellt. \cite[S. 43ff.]{bdbosFAQDigitalfunkBOS2024}

\section{Aktuelle Literaturlage}\label{sec:aktuelle-literaturlage}

Der aktuelle Forschungsstand im Bereich des \gls{TETRA}-Digitalfunks für BOS ist derzeit stark begrenzt.
Einschlägig für diese Arbeit sind insbesondere die von der \gls{BDBOS} veröffentlichten Primärquellen in Form von Dokumente und Webseiten \cite{bdbosFAQDigitalfunkBOS2024,bdbosWarumEsKein2025,bdbosBDBOSGeschichte},
vereinzelt finden sich auch Stellungnahmen oder Dokumentation von bundesweiten Verbänden wie dem \gls{DFV} \cite{agbfdfvErkenntnisseAusGrossflachigen2025} oder von durch die
Länder betriebenen Autorisierten Stellen, welche für die Implementierung und den Betrieb des \gls{BOS}-Digitalfunknetzes
innerhalb ihres Zuständigkeitsbereichs (Land) verantwortlich sind. \cite{zentraldienstderpolizeideslandesbrandenburg20130906Handbuch_EMVU_final_oeffentlich,bosGrundlagenDigitalfunksBOS}

Zusätzlich zu den innerdeutschen Quellen existieren auch einige wenige internationale Publikationen, die sich mit dem \gls{TETRA}-Standard an sich beschäftigen.
Dabei liegt der Fokus neben den technischen Aspekten insbesondere auf den gesundheitlichen Auswirkungen der \gls{TETRA}-Funkstrahlung auf den Menschen. \cite{renImplementationTETRAUsing2011,ozbekTrafficAwareCell2019a,elliottUseTETRAPersonal2019}

Speziell für den Anwendungsfall Deutschlands und der \gls{BOS}, also dem Gegenstand dieser Arbeit, lässt sich jedoch feststellen,
dass nur wenige rein wissenschaftliche Arbeiten zu diesem Thema existieren und die meisten Quellen daher von den primären Akteuren selbst stammen.
Insbesondere durch die Kritikalität des Themas und die sicherheitsrelevanten Aspekte des \gls{BOS}-Digitalfunknetzes
ist davon auszugehen, dass viele Informationen und Erkenntnisse nicht öffentlich zugänglich sind.

Eine folgende Einordnung der Vor- und Nachteile sowie der Risiken und Gefahren des \gls{TETRA}-Digitalfunks ist daher nur anhand der offiziellen Quellen und der wenigen wissenschaftlichen Arbeiten möglich.
Für eine praxisnahe Bewertung und Analyse müssten daher perspektivisch erste empirische Studien und Untersuchungen durchgeführt werden, bspw.\@ in Form von Befragungen oder Interviews mit den Einsatzkräften der \gls{BOS}.

\chapter{Vor- und Nachteile des TETRA-Digitalfunks}\label{ch:vor-und-nachteile}
Die Frage der Vor- und Nachteile, auch in Hinblick auf den Vergleich zum vorherigen analogen Funknetz, ist eine komplexe.
Daher können nicht alle Aspekte in dieser Arbeit betrachtet werden.
Im Folgenden eine essenzielle Auswahl anhand der von der \gls{BDBOS} veröffentlichten Dokumentationen. \cite{bdbosFAQDigitalfunkBOS2024}

\section{Vorteile des TETRA-Digitalfunks}\label{sec:vorteile-tetra-digitalfunk}

\subsection{Gruppenruf, Einzelruf und Priorisierung}\label{subsec:gruppenruf-einzelruf-und-priorisierung}

Ein wesentlicher Vorteil des \gls{TETRA}-Digitalfunks gegenüber konventionellen digitalen Telekommunikationsstandards (wie z.\,B. GSM) ist die Möglichkeit,
sowohl in Gruppen- als auch im Einzelruf zu kommunizieren.
Dabei wird beim Gruppenruf ein Funkspruch simultan an alle Teilnehmer innerhalb einer Rufgruppe (auch \gls{BOS} übergreifend) übertragen.
Diese Gruppen sind jedoch nicht statisch, sondern können dynamisch verwaltet und angepasst werden, so ist zum Beispiel ein Rufgruppenwechsel während eines Einsatzes unkompliziert möglich.
Obendrein bleibt die reine 1:1-Kommunikation (Einzelruf) weiterhin erhalten. \cite[S. 12]{bdbosFAQDigitalfunkBOS2024}

Neben den durch die FwDV 810 \cite{o.a.SprechUndDatenfunkverkehr2018} festgelegten Vorrangstufen für den Funkverkehr, gibt es ebenso darüber hinaus die Möglichkeit, einen Notruf abzusetzen.
Löst ein Teilnehmer diesen aus, so hat er technisch Vorrang gegenüber allen anderen Funksprüchen in der Rufgruppe.
Ebenfalls ist die zeitgleiche Übersendung eines aktuellen bzw\@.
letzt bekannten Standorts möglich. \cite[S. 13]{bdbosFAQDigitalfunkBOS2024}

\subsection{Bundesweite Kommunikation}\label{subsec:bundesweite-kommunikation}

Speziell für den hier betrachteten Anwendungsfall der \gls{BOS} in Deutschland, ermöglicht das von der \gls{BDBOS} betriebene bundeseinheitliche \gls{TETRA}-Netz
eine länderübergreifende Kommunikation zwischen den Einsatzkräften.
Zusätzlich sind auch Grenz- sowie Netzübergreifende Verbindungen möglich, sofern die erforderlichen Schnittstellen implementiert sind. \cite[S. 21, 26]{bdbosFAQDigitalfunkBOS2024}

\subsection{Digitale Datendienste und Short Messages}\label{subsec:digitale-datendienste-und-short-messages}

Neben der klassischen Sprachkommunikation bietet der \gls{TETRA}-Digitalfunk auch die Möglichkeit, digitale Datendienste zu nutzen.
Hierzu zählen unter anderem die Übertragung von kurzen Textnachrichten (Short Messages) auch \gls{SDS} genannt, die Übermittlung von Positionsdaten auf Basis des \gls{GPS} sowie die Nutzung von Statusmeldungen (standardisierte Textnachrichten). \cite[S. 13]{bdbosFAQDigitalfunkBOS2024}
Weiterführend sind auch digitale Alarmierungen von Einsatzkräften über den \gls{TETRA}-Digitalfunk möglich. \cite[S. 23]{bdbosFAQDigitalfunkBOS2024}

\subsection{Abhör- und Informationssicherheit}\label{subsec:abhoer-und-informationssicherheit}

Den wohl größten Vorteil bietet die hohe Abhör- und Informationssicherheit des \gls{BOS}-Digitalfunks.
Wie bereits in \autoref{subsec:ende-zu-ende-verschlusselung} beschrieben, wird eine Ende-zu-Ende-Verschlüsselung der Kommunikation zwischen den Endgeräten innerhalb
des \gls{BOS}-Netz realisiert.
Im Gegensatz zum analogen Funk ist somit das Abhören der Kommunikation durch Unbefugte erheblich erschwert.
Hierfür liegen unter anderem auch erste Zertifizierungen nach ISO 27001 vor. \cite[S. 43]{bdbosFAQDigitalfunkBOS2024}

\section{Nachteile des TETRA-Digitalfunks}\label{sec:nachteile-tetra-digitalfunk}

\subsection{Netzausfälle und Störanfälligkeit}\label{subsec:netzausfaelle-und-stoeranfalligkeit}

Trotz der mittlerweile jahrelangen Betriebszeit des \gls{BOS}-Digitalfunknetzes, kommt es immer wieder zu flächendeckenden Ausfällen.
Zuletzt am \date{06. Mai 2025} kam es zu einem großflächigen Ausfall des Netzes, welcher mehrere Stunden andauerte und deutschlandweit die Kommunikation der \gls{BOS} beeinträchtigte. \cite[S. 1]{agbfdfvErkenntnisseAusGrossflachigen2025}

Auch wenn ein externer Angriff ausgeschlossen werden konnte, zeigt dieser Vorfall die Anfälligkeit des Netzes.
Grund für den Ausfall war eine Fehlfunktion in der Kernnetz-Infrastruktur der \gls{BDBOS}, welchem ein Softwarefehler des Herstellers vorausging. \cite[S. 3]{agbfdfvErkenntnisseAusGrossflachigen2025}
Nichtsdestotrotz verdeutlicht dieser Vorfall, dass es auch fremden Angreifern möglich sein könnte, das Netz an bestimmten anfälligen Stellen zu stören.

Zwar existieren Rückfallebenen wie autarke Basisstationen, die den Funkverkehr innerhalb einer Funkzelle auch bei Ausfall des Kernnetzes ermöglichen,
sowie der Direktmodus (\gls{DMO}), der die Kommunikation zwischen Endgeräten ohne Netzinfrastruktur erlaubt,
dennoch sind diese nur bedingt praktikabel und stellen keine vollwertige Alternative zum regulären Netzbetrieb dar. \cite[S. 3f.]{agbfdfvErkenntnisseAusGrossflachigen2025}

\subsection{Benutzerfreundlichkeit und Bedienung}\label{subsec:benutzerfreundlichkeit-und-bedienung}

Die Bedienung der \gls{TETRA}-Endgeräte ist im Vergleich zu analogen Funkgeräten komplexer.
Dies liegt zum Einen an der Vielzahl an Funktionen und Möglichkeiten (wie oben beschrieben), die die Endgeräte bieten.
Zum Anderen erfordert die grundlegende Umstellung von analogem auf digitalen Funk eine gewisse Einarbeitungszeit insbesondere für ältere Einsatzkräfte.

Aus technischer Perspektive bildet der Rufaufbau den wohl größten Kritikpunkt.
Im Gegensatz zum analogen Funk, bei dem ein Funkspruch unmittelbar übertragen wird, benötigt der \gls{TETRA}-Digitalfunk
eine gewisse Zeit für den Verbindungsaufbau, um bspw\@.
alle Endgeräte innerhalb der Rufgruppe zu initialisieren.
Auch die Verschlüsselung der Kommunikation trägt zu dieser Verzögerung bei.

\chapter{Fazit und Ausblick}\label{ch:fazit-und-ausblick}

Mit der Einführung des \gls{TETRA}-Digitalfunks hat die \gls{BOS} in Deutschland einen bedeutenden Schritt in Richtung moderner und sicherer Kommunikation gemacht.
Insbesondere die sichere Informationsübertragung und die flexiblen Möglichkeiten in der Kommunikation stellen klare Vorteile dar.
Allerdings zeigen auch die Ausfälle der letzten Jahre und Monate, dass das Netz, welches durch die \gls{BDBOS} betrieben wird, noch nicht vollständig ausgereift ist,
trotz der mittlerweile jahrelangen Betriebszeit.

Darüber hinaus sollte auch betrachtet werden, ob der \gls{TETRA}-Standard in Zukunft noch den Anforderungen der \gls{BOS} gerecht wird,
insbesondere im Hinblick auf die steigenden Anforderungen an Datenübertragungsraten und moderne Kommunikationsdienste.
Schließlich ist dieser Standard in seiner aktuellen Form seit 2001 unverändert im Einsatz.

Nachholbedarf besteht daher insbesondere in der Stabilität und Zuverlässigkeit des Netzes.
Der Ausbau der Netzabdeckung, die Schulung der Einsatzkräfte im Umgang mit den Endgeräten sowie die kontinuierliche Weiterentwicklung des Netzes sind essenzielle Schritte,
um den Anforderungen der \gls{BOS} auch in Zukunft gerecht zu werden.
Hier sind vor allem die \gls{BDBOS} und die zuständigen Autorisierten Stellen in den Ländern gefordert.

%Diese Einführung soll einen kurzen Überblick über die allgemeinen Möglichkeiten von \LaTeX{} geben.
%
%Es kann auf Bilder wie das HWR-Logo verwiesen werden (s. \autoref{fig:hwrlogo}) oder auf Tabellen (s. \autoref{table:tab_spalten}).
%
%\begin{figure}[h]
%  \centering
%  \includegraphics[scale=0.8]{bilder/header_logo.png}
%  \label{fig:hwrlogo}       %fig:ID
%  \caption[HWR-Logo: Überschrift Abbildungsverzeichnis]{HWR-Logo Bildunterschrift}    %Bildunterschrift
%\end{figure}
%
%\begin{table}[h]
%\centering
%\small
%\begin{tabular}{|l l l l|}
%\hline
%Spalte 1 & Spalte 2 & Spalte 3 & Spalte 4\\
%\hline
%\end{tabular}
%\caption{Eine Tabelle mit Spalten}
%\label{table:tab_spalten}
%\end{table}
%
%Auch Quellenverweise sind möglich. Quellen werden in der Datei literatur.bib angelegt und tauchen automatisch im Literaturverzeichnis auf, wenn der Text einen entsprechenden Verweis enthält \cite[S. 42-1337]{XP}. Auch Glossareinträge wie \gls{glos:AntwD} oder Abkürzungen wie \gls{DMZ}, \gls{AD} und \gls{CD} folgen dieser Regel.
%
%Verweise auf Kapitel sind ebenfalls möglich. Kapitel werden zu diesem Zweck mit einem Label versehen (s. \autoref{sec:allgemein}).
%Außerdem gibt es natürlich so schöne Dinge wie Aufzählungen\footnote{und Fußnoten}
%
%\begin{itemize}
%	\item Wenn bloß eine Aufzählung
%	\item benötigt wird
%\end{itemize}
%
%und Numerierungen
%
%\begin{enumerate}
%	\item Wenn eine Numerierung
%	\item gewünscht ist
%\end{enumerate}
%
%\section{Besonderheit}
%
%Unter dem vorliegenden Kapitel ist eine Besonderheit dieser Vorlage aufgeführt. Aufgrund von Platz- und Übersichtsgründen soll wie in der ursprünglichen Vorlage im Text nur ein abgekürzter Literatureintrag angezeigt werden, wie \cite[S. 42-1337]{XP}, aber im Literaturverzeichnis soll sich der Gesamteintrag weitestgehend an den APA-Richtlinien orientieren.
%
%Die Formatierung des Literaturverzeichnisses weicht daher vom Standard ab. Die im Paket enthaltene Datei ,,hwrbib.bst'' bietet diese Möglichkeit an.
%
%Hier noch der Trigger für einige Literaturverzeichniseinträge:
%
%\begin{itemize}
%	\item \cite{HUMMWIE_2005}
%	\item \cite{DWD_Beau}
%	\item \cite{CRISP}
%\end{itemize}

%%%%%%%%%%%%%%%%%%%%%%%%%%
% Quellen
%%%%%%%%%%%%%%%%%%%%%%%%%

\bibliography{literatur}

\bibliographystyle{ieeetr}
%% \bibliographystyle{alpha} %% tu es nicht, niemals, das ist eklig, nicht einkommentieren

\chapter*{Ehrenwörtliche Erklärung}
\addcontentsline{toc}{chapter}{Ehrenwörtliche Erklärung}

% Keine Kopf- und Fußzeilen ausgeben
\thispagestyle{empty}
% Aber trotzdem ins Inhaltsverzeichnis aufnehmen
%\addcontentsline{toc}{section}{Eidesstattliche Erklärung}

% Hier der offizielle Text der eidesstattlichen Erklärung
Hiermit versichere ich, dass ich die vorliegende Arbeit in
allen Teilen selbstständig angefertigt und keine anderen als die in der Arbeit
angegebenen Quellen und Hilfsmittel benutzt habe, und dass die Arbeit in
gleicher oder ähnlicher Form in noch keiner anderen Prüfung vorgelegen hat.
Sämtliche wörtlichen oder sinngemäßen Übernahmen und Zitate, sowie alle
Abschnitte, die mithilfe von KI-basierten Tools entworfen, verfasst und/oder
bearbeitet wurden, sind kenntlich gemacht und nachgewiesen.

Im Anhang meiner Arbeit habe ich sämtliche KI-basierte
Hilfsmittel angegeben. Diese sind mit Produktnamen und formulierten Eingaben
(Prompts) in einem KI-Verzeichnis ausgewiesen.

Ich bin mir bewusst, dass die Verwendung von Texten oder
anderen Inhalten und Produkten, die durch KI-basierte Tools generiert wurden,
keine Garantie für deren Qualität darstellt. Ich verantworte die Übernahme
jeglicher von mir verwendeter maschinell generierter Passagen vollumfänglich
selbst und trage die Verantwortung für eventuell durch die KI generierte
fehlerhafte oder verzerrte Inhalte, fehlerhafte Referenzen, Verstöße gegen das
Datenschutz- und Urheberrecht oder Plagiate.

% Etwas Abstand für die Unterschrift
\vspace{2cm}

% Hier kommt die Unterschrift drüber
\begin{tabular}{lp{4em}l} 
 \hspace{5cm}   && \hspace{4cm} \\\cline{1-1}\cline{3-3} 
 Ort, Datum     && \studentName
\end{tabular}

\end{document}

