Die Frage der Vor- und Nachteile, auch in Hinblick auf den Vergleich zum vorherigen analogen Funknetz, ist eine komplexe.
Daher können nicht alle Aspekte in dieser Arbeit betrachtet werden.
Im Folgenden eine essenzielle Auswahl anhand der von der \gls{BDBOS} veröffentlichten Dokumentationen. \cite{bdbosFAQDigitalfunkBOS2024}

\section{Vorteile des TETRA-Digitalfunks}\label{sec:vorteile-tetra-digitalfunk}

\subsection{Gruppenruf, Einzelruf und Priorisierung}\label{subsec:gruppenruf-einzelruf-und-priorisierung}

Ein wesentlicher Vorteil des \gls{TETRA}-Digitalfunks gegenüber konventionellen digitalen Telekommunikationsstandards (wie z.\,B. GSM) ist die Möglichkeit,
sowohl in Gruppen- als auch im Einzelruf zu kommunizieren.
Dabei wird beim Gruppenruf ein Funkspruch simultan an alle Teilnehmer innerhalb einer Rufgruppe (auch \gls{BOS} übergreifend) übertragen.
Diese Gruppen sind jedoch nicht statisch, sondern können dynamisch verwaltet und angepasst werden, so ist zum Beispiel ein Rufgruppenwechsel während eines Einsatzes unkompliziert möglich.
Obendrein bleibt die reine 1:1-Kommunikation (Einzelruf) weiterhin erhalten. \cite[S. 12]{bdbosFAQDigitalfunkBOS2024}

Neben den durch die FwDV 810 \cite{o.a.SprechUndDatenfunkverkehr2018} festgelegten Vorrangstufen für den Funkverkehr, gibt es ebenso darüber hinaus die Möglichkeit, einen Notruf abzusetzen.
Löst ein Teilnehmer diesen aus, so hat er technisch Vorrang gegenüber allen anderen Funksprüchen in der Rufgruppe.
Ebenfalls ist die zeitgleiche Übersendung eines aktuellen bzw\@.
letzt bekannten Standorts möglich. \cite[S. 13]{bdbosFAQDigitalfunkBOS2024}

\subsection{Bundesweite Kommunikation}\label{subsec:bundesweite-kommunikation}

Speziell für den hier betrachteten Anwendungsfall der \gls{BOS} in Deutschland, ermöglicht das von der \gls{BDBOS} betriebene bundeseinheitliche \gls{TETRA}-Netz
eine länderübergreifende Kommunikation zwischen den Einsatzkräften.
Zusätzlich sind auch Grenz- sowie Netzübergreifende Verbindungen möglich, sofern die erforderlichen Schnittstellen implementiert sind. \cite[S. 21, 26]{bdbosFAQDigitalfunkBOS2024}

\subsection{Digitale Datendienste und Short Messages}\label{subsec:digitale-datendienste-und-short-messages}

Neben der klassischen Sprachkommunikation bietet der \gls{TETRA}-Digitalfunk auch die Möglichkeit, digitale Datendienste zu nutzen.
Hierzu zählen unter anderem die Übertragung von kurzen Textnachrichten (Short Messages) auch \gls{SDS} genannt, die Übermittlung von Positionsdaten auf Basis des \gls{GPS} sowie die Nutzung von Statusmeldungen (standardisierte Textnachrichten). \cite[S. 13]{bdbosFAQDigitalfunkBOS2024}
Weiterführend sind auch digitale Alarmierungen von Einsatzkräften über den \gls{TETRA}-Digitalfunk möglich. \cite[S. 23]{bdbosFAQDigitalfunkBOS2024}

\subsection{Abhör- und Informationssicherheit}\label{subsec:abhoer-und-informationssicherheit}

Den wohl größten Vorteil bietet die hohe Abhör- und Informationssicherheit des \gls{BOS}-Digitalfunks.
Wie bereits in \autoref{subsec:ende-zu-ende-verschlusselung} beschrieben, wird eine Ende-zu-Ende-Verschlüsselung der Kommunikation zwischen den Endgeräten innerhalb
des \gls{BOS}-Netz realisiert.
Im Gegensatz zum analogen Funk ist somit das Abhören der Kommunikation durch Unbefugte erheblich erschwert.
Hierfür liegen unter anderem auch erste Zertifizierungen nach ISO 27001 vor. \cite[S. 43]{bdbosFAQDigitalfunkBOS2024}

\section{Nachteile des TETRA-Digitalfunks}\label{sec:nachteile-tetra-digitalfunk}

\subsection{Netzausfälle und Störanfälligkeit}\label{subsec:netzausfaelle-und-stoeranfalligkeit}

Trotz der mittlerweile jahrelangen Betriebszeit des \gls{BOS}-Digitalfunknetzes, kommt es immer wieder zu flächendeckenden Ausfällen.
Zuletzt am \date{06. Mai 2025} kam es zu einem großflächigen Ausfall des Netzes, welcher mehrere Stunden andauerte und deutschlandweit die Kommunikation der \gls{BOS} beeinträchtigte. \cite[S. 1]{agbfdfvErkenntnisseAusGrossflachigen2025}

Auch wenn ein externer Angriff ausgeschlossen werden konnte, zeigt dieser Vorfall die Anfälligkeit des Netzes.
Grund für den Ausfall war eine Fehlfunktion in der Kernnetz-Infrastruktur der \gls{BDBOS}, welchem ein Softwarefehler des Herstellers vorausging. \cite[S. 3]{agbfdfvErkenntnisseAusGrossflachigen2025}
Nichtsdestotrotz verdeutlicht dieser Vorfall, dass es auch fremden Angreifern möglich sein könnte, das Netz an bestimmten anfälligen Stellen zu stören.

Zwar existieren Rückfallebenen wie autarke Basisstationen, die den Funkverkehr innerhalb einer Funkzelle auch bei Ausfall des Kernnetzes ermöglichen,
sowie der Direktmodus (\gls{DMO}), der die Kommunikation zwischen Endgeräten ohne Netzinfrastruktur erlaubt,
dennoch sind diese nur bedingt praktikabel und stellen keine vollwertige Alternative zum regulären Netzbetrieb dar. \cite[S. 3f.]{agbfdfvErkenntnisseAusGrossflachigen2025}

\subsection{Benutzerfreundlichkeit und Bedienung}\label{subsec:benutzerfreundlichkeit-und-bedienung}

Die Bedienung der \gls{TETRA}-Endgeräte ist im Vergleich zu analogen Funkgeräten komplexer.
Dies liegt zum Einen an der Vielzahl an Funktionen und Möglichkeiten (wie oben beschrieben), die die Endgeräte bieten.
Zum Anderen erfordert die grundlegende Umstellung von analogem auf digitalen Funk eine gewisse Einarbeitungszeit insbesondere für ältere Einsatzkräfte.

Aus technischer Perspektive bildet der Rufaufbau den wohl größten Kritikpunkt.
Im Gegensatz zum analogen Funk, bei dem ein Funkspruch unmittelbar übertragen wird, benötigt der \gls{TETRA}-Digitalfunk
eine gewisse Zeit für den Verbindungsaufbau, um bspw\@.
alle Endgeräte innerhalb der Rufgruppe zu initialisieren.
Auch die Verschlüsselung der Kommunikation trägt zu dieser Verzögerung bei.